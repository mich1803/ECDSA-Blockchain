\documentclass[aspectratio=169]{beamer}

\usetheme{Madrid}
\usecolortheme{default}
\setbeamertemplate{navigation symbols}{}

\usepackage[T1]{fontenc}
\usepackage[utf8]{inputenc}
\usepackage[english]{babel}
\usepackage{amsmath,amssymb,mathtools}
\usepackage{booktabs}
\usepackage{listings}
\usepackage{xcolor}
\usepackage{tikz}
\usepackage{pgfplots}
\usepackage{fontawesome5}
\usepackage{qrcode}


\pgfplotsset{compat=1.18}

\setbeamercolor{block title alerted}{fg=white,bg=cyan!70}
\setbeamercolor{block body alerted}{fg=black,bg=cyan!10}

% --- Gray theme for exercise slides ---
\definecolor{exercisegray}{RGB}{90,90,90}
\definecolor{exercisegraylight}{RGB}{235,235,235}

\newenvironment{exerciseFrame}[1]
{
  % Save current colors
  \begingroup
  \setbeamercolor{frametitle}{fg=white,bg=exercisegray}
  \setbeamercolor{structure}{fg=exercisegray}
  \setbeamercolor{itemize item}{fg=exercisegray}
  \setbeamercolor{block title}{fg=white,bg=exercisegray}
  \setbeamercolor{block body}{fg=black,bg=exercisegraylight}

  \begin{frame}{#1}
}
{
  \end{frame}
  \endgroup
}


\title[Educational Blockchain Implementation]{%
Educational Blockchain Implementation: \\
Structure, Experiments, and Possible Attacks
}

\author[M.~Magrini \,\textbullet\, F.~Marrocco \,\textbullet\, L.~V.~Petrarca]{%
Magrini Michele, Marrocco Francesco, Leo Vincenzo Petrarca \\ \quad \\ Course: Institutions of Algebra and Geometry \\ Master's Degree in Applied Mathematics \\ Sapienza University of Rome
}

\date{11 Feb 2026}

\definecolor{codebg}{RGB}{248,248,248}
\lstset{
  basicstyle=\ttfamily\small,
  backgroundcolor=\color{codebg},
  frame=single,
  breaklines=true,
  columns=fullflexible
}

\begin{document}

\begin{frame}
  \titlepage
\end{frame}

\begin{frame}{Presentation goals}

\vspace{0.5em}
\textbf{Outline:}
\begin{enumerate}
  \item Introduction and initial definitions.
  \item Experiment \emph{Transactions} (wallets, genesis, nodes, tx, mining, state).
  \item Experiment \emph{Weak Nonce Reuse Attack} (setup, vulnerability, exploit, key recovery).
\end{enumerate}
\end{frame}

\section{Introduction and initial definitions}

\begin{frame}{Curve definition and wallets}

% ---------- TOP PART: COLUMNS ----------
\begin{columns}[T]

\begin{column}{0.65\textwidth}
\begin{itemize}
  \item Finite field $\mathbb{F}_p$ with secp256k1 curve\footnotemark[1]:
  \[
    E: y^2 \equiv x^3 + 7 \pmod p.
  \]
  \item Group of points with generator $G$ and prime order $n$\footnotemark[2]
  \item Private key: $d \in \{1,\dots,n-1\}$.
  \item Public key: $Q = dG$.
  \item Address: last 20 bytes of $\mathrm{Keccak256}$.
  \item Transaction signed with ECDSA and SHA-256 digest\footnotemark[3] of the JSON payload\footnotemark[4]
\end{itemize}
\end{column}

\begin{column}{0.35\textwidth}
\centering
\begin{tikzpicture}[scale=0.85]
\begin{axis}[
  axis lines=middle,
  xtick=\empty,
  ytick=\empty,
  width=\linewidth,
  height=0.65\textheight,
  domain=-3.5:3.5,
  samples=200
]
\addplot[blue, thick] {sqrt(x^3 + 7)};
\addplot[blue, thick] {-sqrt(x^3 + 7)};
\end{axis}
\end{tikzpicture}

{\scriptsize Real curve $y^2 = x^3 + 7$}
\end{column}

\end{columns}

% ---------- SPACE ----------
\vfill

% ---------- LINE ----------
\vspace{0.35em}

% ---------- FOOTNOTE FULL WIDTH ----------
\scriptsize
\footnotetext[1]{$p=\texttt{0xFFFFFFFFFFFFFFFFFFFFFFFFFFFFFFFFFFFFFFFFFFFFFFFFFFFFFFFEFFFFFC2F}$.}
\footnotetext[2]{$n=\texttt{0xFFFFFFFFFFFFFFFFFFFFFFFFFFFFFFFEBAAEDCE6AF48A03BBFD25E8CD0364141}$.}
\footnotetext[3]{Keccak256 and SHA-256 are hash functions that map arbitrary input to a fixed 256-bit output.}
\footnotetext[4]{The payload is the informative content of the transaction, i.e., what is cryptographically authenticated.}

\end{frame}


\section{Experiment 1: Transactions}

\begin{frame}{Wallet generation and initial state}

\begin{enumerate}
  \item We generate a wallet for each user.
  \begin{itemize}
      \item Each wallet is a pair $(d,Q)$ plus an address identifier.
      \item Possession of $d$ enables signing and thus logical control of the funds tied to the address.
  \end{itemize}
  
  \item We define the initial state $\Sigma_0: \text{address} \mapsto (balance, nonce)$.
  \begin{itemize}
      \item It is the shared reference for deterministic validation across nodes.
  \end{itemize}
\end{enumerate}

\vfill
\begin{center}
\begin{minipage}{0.7\textwidth}
\begin{alertblock}{\textbf{\faLaptop\ Implementation}}
Wallet generation and initial state
\end{alertblock}
\end{minipage}
\end{center}
\end{frame}



\begin{exerciseFrame}{Example of wallet creation over $\mathbb{F}_{17}$ [E1]}

\small 
\textbf{Elliptic curve} $E:\; y^2 \equiv x^3 + 2x + 2 \pmod{17}.$

\textbf{Group} $E(\mathbb{F}_{17}) \text{ has } 19 \text{ points (including } \mathcal{O}\text{)} \Rightarrow n=19.$

\textbf{Chosen generator:} $G=(0,6), \quad \operatorname{ord}(G)=19.$

\textbf{Keys:} $d=7,\quad Q=dG=7G.$

\vspace{0.2em}
\hrule
\vspace{0.3em}

Sum and Difference are defined as follows:

For $P\neq Q$:
$
\lambda=\frac{y_Q-y_P}{x_Q-x_P}\ (\mathrm{mod}\ 17),\quad
x_{P+Q}=\lambda^2-x_P-x_Q,\quad
y_{P+Q}=\lambda(x_P-x_{P+Q})-y_P.
$

For $2P$: $\lambda=\frac{3x_P^2+a}{2y_P}\ (\mathrm{mod}\ 17).$

\vspace{0.3em}
\hrule
\vspace{0.2em}

Computation of $2G$: 
$\lambda=\frac{3\cdot 0^2+2}{2\cdot 6}=\frac{2}{12}\equiv 2\cdot 12^{-1}\equiv 2\cdot 10\equiv 3\ (\mathrm{mod}\ 17)$

$x_{2G}=3^2-0-0=9,\qquad y_{2G}=3(0-9)-6=-33\equiv 1\ (\mathrm{mod}\ 17) \to 2G=(9,1).$


Iterating the sum we obtain: \quad $7G=(13,10)=Q$.

\vspace{0.3em}
\hrule
\vspace{0.3em}

We define a \textbf{didactic hash function}:
$H(x,y)=x+2y\ (\mathrm{mod}\ 17),\qquad $

We obtain $H(Q)=H(13,10)=13+2\cdot 10=33\equiv 16.$

\vspace{0.3em}
\hrule
\vspace{0.3em}

We have therefore generated our \textbf{wallet:}\quad $(d,Q,H(Q))=\bigl(7,(13,10),16\bigr)$.
\end{exerciseFrame}



\begin{frame}{Three nonces to distinguish}
\begin{block}{Cryptographic nonce $k$}
Used in ECDSA signatures. \textbf{If reused}, it can expose $d$.
\end{block}
\begin{block}{Account nonce $n_{acc}$}
Per-account counter in the blockchain (anti-replay / ordering of tx).
\end{block}
\begin{block}{Mining nonce $n_{pow}$}
Variable in the block to satisfy PoW: hash with leading zeros.
\end{block}
\end{frame}

\begin{frame}{How is a transaction signed with ECDSA?}
Let $z$ be the message digest reduced mod $n$, we obtain the signature as follows:
\[
R = kG, \quad r = x_R \bmod n, \quad
s = k^{-1}(z + rd) \bmod n.
\]

Verification checks that the point reconstructed by combining message and public key
reproduces the same $r$ value produced by the signer.
\[
u_1 = zs^{-1},\; u_2 = rs^{-1},\; X = u_1G + u_2Q
\]
where $X$ is the point reconstructed in verification that,
for a valid signature, coincides with $R = kG$.

\vfill
\hrule
\vspace{.45em}

Security depends critically on the secrecy and uniqueness of the cryptographic nonce ($k$). We will see a possible attack at the end if this nonce is duplicated.
\end{frame}

\begin{frame}{How is a block mined?}
\textbf{Mathematical / protocol side:}
\begin{itemize}
  \item Build the block header with previous hash, tx, difficulty, proposer.
  \item The \textbf{difficulty} (default $d=2$ in the project) imposes how many initial hex zeros the block hash must have.
  \item Search for $n_{pow}$ such that:
  \[
    \mathrm{H_{SHA}}(\text{header}, n_{pow}) = h, \quad h \text{ starts with } d \text{ leading hex zeros.}
  \]
  \item Success probability per attempt: $\approx 16^{-d}$ (uniform hash).
  \item Expected attempts: $\approx 16^d$ (with $d=2$, about $256$ average tries).
  \item Once found, the block is a valid PoW candidate.
\end{itemize}

\end{frame}


\begin{frame}{Transaction creation and mining}
\textbf{Conceptual pipeline:}
\begin{enumerate}
  \item Build canonical payload (JSON).
  \item Compute digest $z = H(payload)$.
  \item Compute ECDSA signature $\Rightarrow (r,s)$.
  \item \textbf{Cryptographic authenticity}: node verifies signature + rules (address, funds, nonce).
  \item \textbf{Protocol validity}: if valid, tx enters mempool and is propagated to peers.
  \item The miner takes the whole block and searches for $n_{pow}$ that satisfies the condition. If successful, it is rewarded.
\end{enumerate}

\vfill
\begin{center}
\begin{minipage}{0.7\textwidth}
\begin{alertblock}{\textbf{\faLaptop\ Implementation}}
Sending transactions and mining
\end{alertblock}
\end{minipage}
\end{center}

\end{frame}

\begin{exerciseFrame}{Example of ECDSA signature [E2]}

\small

\textbf{Setup (as in E1):}


$E:\; y^2 \equiv x^3 + 2x + 2 \pmod{17}, \quad G=(0,6), \quad n=19$

$d=7, \quad Q=(13,10)$

\vspace{0.3em}
\hrule
\vspace{0.3em}

\textbf{Message:} $z = 11 \in \mathbb{Z}_{19}$ (transaction digest)

\textbf{Choice of cryptographic nonce:} $k=5$

\vspace{0.3em}
\hrule
\vspace{0.3em}

\textbf{Computation:}\\
\[
R = kG = 5G = (10,11) \to r = x_R \bmod n = 10
\]

\[
k^{-1} \equiv 5^{-1} \equiv 4 \pmod{19}
\]

\[
s = k^{-1}(z + rd) \bmod n = 4(11 + 10\cdot 7) \bmod 19 = 1
\]

\vspace{0.3em}
\hrule
\vspace{0.3em}

Thus we have the \textbf{ECDSA signature:} $(r,s) = (10,1)$



\end{exerciseFrame}



\section{Experiment 2: Weak Nonce Reuse Attack}

\begin{frame}{Mathematical problem: reusing the same $k$}
For two different messages $z_1, z_2$ signed with the same $k$:
\[
s_1 = k^{-1}(z_1 + rd), \qquad s_2 = k^{-1}(z_2 + rd).
\]
Subtracting:
\[
k = (z_1 - z_2)(s_1 - s_2)^{-1} \bmod n.
\]
Then:
\[
d = (s_1k - z_1)r^{-1} \bmod n.
\]
\textbf{Conclusion:} two signatures with the same $r$ (thus the same $k$) are enough to recover the private key.
\end{frame}

\begin{frame}{Attack on a weak transaction}
\textbf{Vulnerable pipeline:}
\begin{enumerate}
  \item Generate a random $k_0$ once.
  \item Create tx1 with current account nonce, sign with $k_0$, send.
  \item Mine a block immediately (tx1 confirmed, account nonce incremented).
  \item Create tx2 with \textbf{new account nonce} but same $k_0$, send.
\end{enumerate}

\textbf{Attacker pipeline:}
\begin{enumerate}
  \item Read chain state + mempool via node API.
  \item Scan all tx and group by pair (pubkey, $r$).
  \item If two tx with the same $r$ are found, apply recovery formulas for $k$ and $d$.
\end{enumerate}

\vfill
\begin{center}
\begin{minipage}{0.7\textwidth}
\begin{alertblock}{\textbf{\faLaptop\ Implementation}}
Attack on a weak transaction
\end{alertblock}
\end{minipage}
\end{center}

\end{frame}

\begin{exerciseFrame}{Example of ECDSA nonce reuse (E3)}

\small

\textbf{Scenario (as in E1 and E2):}
\[
E:\; y^2 \equiv x^3 + 2x + 2 \pmod{17}, \qquad
G=(0,6), \qquad n=19
\]
\[
d=7, \qquad Q=(13,10)
\]

\textbf{First transaction:}
\[
z_1=11, \quad k=5 \quad \Longrightarrow \quad (r,s_1)=(10,1)
\]

\textbf{Second transaction (duplicate nonce):}
\[
z_2=3, \quad k=5 \quad \Longrightarrow \quad (r,s_2)=(10,7)
\]

\hrule
\vspace{0.4em}

\textbf{Nonce recovery:}
\[
k = (z_1 - z_2)(s_1 - s_2)^{-1} \bmod n
\]
\[
z_1-z_2 = 8, \quad s_1-s_2 = -6 \equiv 13, \quad 13^{-1}\equiv 3
\]
\[
k = 8 \cdot 3 \bmod 19 = 5
\]

\textbf{Private key recovery:}
\[
d = (s_1k - z_1) r^{-1} \bmod n
\]
\[
s_1k - z_1 = 5 - 11 = -6 \equiv 13, \quad r^{-1} = 10^{-1} \equiv 2
\]
\[
d = 13 \cdot 2 \bmod 19 = 7
\]

\begin{alertblock}{Conclusion}
Reusing the nonce $k$ in ECDSA enables the \textbf{complete recovery of the private key}.
\end{alertblock}

\end{exerciseFrame}

\begin{frame}{Conclusion}

\vfill

\begin{center}
{\LARGE \textbf{Thank you for your attention}}

\vspace{0.5cm}

{\large
Michele Magrini \hspace{1em} (matr. \texttt{2066963})\\[0.4em]
Francesco Marrocco \hspace{1em} (matr. \texttt{2062223})\\[0.4em]
Leo Vincenzo Petrarca \hspace{1em} (matr. \texttt{2087113})
}


\vspace{0.6cm}

\qrcode[height=3cm]{https://github.com/mich1803/ECDSA-Blockchain}

\vspace{0.2cm}

\textbf{\href{https://github.com/mich1803/ECDSA-Blockchain}{GitHub Repository}}

\end{center}

\vfill

\end{frame}



\end{document}
