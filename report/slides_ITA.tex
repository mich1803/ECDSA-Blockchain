\documentclass[aspectratio=169]{beamer}

\usetheme{Madrid}
\usecolortheme{default}
\setbeamertemplate{navigation symbols}{}

\usepackage[T1]{fontenc}
\usepackage[utf8]{inputenc}
\usepackage[italian]{babel}
\usepackage{amsmath,amssymb,mathtools}
\usepackage{booktabs}
\usepackage{listings}
\usepackage{xcolor}
\usepackage{tikz}
\usepackage{pgfplots}
\usepackage{fontawesome5}
\usepackage{qrcode}


\pgfplotsset{compat=1.18}

\setbeamercolor{block title alerted}{fg=white,bg=cyan!70}
\setbeamercolor{block body alerted}{fg=black,bg=cyan!10}

% --- Tema grigio per slide esercizi ---
\definecolor{exercisegray}{RGB}{90,90,90}
\definecolor{exercisegraylight}{RGB}{235,235,235}

\newenvironment{exerciseFrame}[1]
{
  % Salva colori correnti
  \begingroup
  \setbeamercolor{frametitle}{fg=white,bg=exercisegray}
  \setbeamercolor{structure}{fg=exercisegray}
  \setbeamercolor{itemize item}{fg=exercisegray}
  \setbeamercolor{block title}{fg=white,bg=exercisegray}
  \setbeamercolor{block body}{fg=black,bg=exercisegraylight}

  \begin{frame}{#1}
}
{
  \end{frame}
  \endgroup
}


\title[Implementazione di una Blockchain Didattica]{%
Implementazione di una Blockchain Didattica: \\
Struttura, Esperimenti e Possibili Attacchi
}

\author[M.~Magrini \,\textbullet\, F.~Marrocco \,\textbullet\, L.~V.~Petrarca]{%
Magrini Michele, Marrocco Francesco, Leo Vincenzo Petrarca \\ \quad \\ Corso di Istituzioni di Algebra e Geometria \\ Laurea Magistrale in Matematica Applicata \\ Università di Roma "La Sapienza"
}

\date{11 Feb 2026}

\definecolor{codebg}{RGB}{248,248,248}
\lstset{
  basicstyle=\ttfamily\small,
  backgroundcolor=\color{codebg},
  frame=single,
  breaklines=true,
  columns=fullflexible
}

\begin{document}

\begin{frame}
  \titlepage
\end{frame}

\begin{frame}{Obiettivo della presentazione}

\vspace{0.5em}
\textbf{Indice:}
\begin{enumerate}
  \item Introduzione e definizioni iniziali.
  \item Esperimento \emph{Transazioni} (wallet, genesis, nodi, tx, mining, state).
  \item Esperimento \emph{Attacco Weak Nonce Reuse} (setup, vulnerabilità, exploit, recovery key).
\end{enumerate}
\end{frame}

\section{Introduzione e definizioni iniziali}

\begin{frame}{Definizione della Curva e dei Wallet}

% ---------- PARTE ALTA: COLONNE ----------
\begin{columns}[T]

\begin{column}{0.65\textwidth}
\begin{itemize}
  \item Campo finito $\mathbb{F}_p$ con curva secp256k1\footnotemark[1]:
  \[
    E: y^2 \equiv x^3 + 7 \pmod p.
  \]
  \item Gruppo dei punti con generatore $G$ e ordine primo $n$\footnotemark[2]
  \item Chiave privata: $d \in \{1,\dots,n-1\}$.
  \item Chiave pubblica: $Q = dG$.
  \item Indirizzo: ultimi 20 byte di $\mathrm{Keccak256}$.
  \item Transazione firmata con ECDSA con digest SHA-256\footnotemark[3] sul payload\footnotemark[4] JSON della transazione
\end{itemize}
\end{column}

\begin{column}{0.35\textwidth}
\centering
\begin{tikzpicture}[scale=0.85]
\begin{axis}[
  axis lines=middle,
  xtick=\empty,
  ytick=\empty,
  width=\linewidth,
  height=0.65\textheight,
  domain=-3.5:3.5,
  samples=200
]
\addplot[blue, thick] {sqrt(x^3 + 7)};
\addplot[blue, thick] {-sqrt(x^3 + 7)};
\end{axis}
\end{tikzpicture}

{\scriptsize Curva reale $y^2 = x^3 + 7$}
\end{column}

\end{columns}

% ---------- SPAZIO ----------
\vfill

% ---------- LINEA ----------
\vspace{0.35em}

% ---------- FOOTNOTE A TUTTA LARGHEZZA ----------
\scriptsize
\footnotetext[1]{$p=\texttt{0xFFFFFFFFFFFFFFFFFFFFFFFFFFFFFFFFFFFFFFFFFFFFFFFFFFFFFFFEFFFFFC2F}$.}
\footnotetext[2]{$n=\texttt{0xFFFFFFFFFFFFFFFFFFFFFFFFFFFFFFFEBAAEDCE6AF48A03BBFD25E8CD0364141}$.}
\footnotetext[3]{Keccak256 e SHA-256 sono funzioni hash che mappano input arbitrario in un output fisso da 256 bit.}
\footnotetext[4]{Il payload è il contenuto informativo della transazione, cioè ciò che si vuole autenticare crittograficamente.}

\end{frame}


\section{Esperimento 1: Transazioni}

\begin{frame}{Generazione dei Wallet e dello Stato Iniziale}

\begin{enumerate}
  \item Generiamo un wallet per ogni utente.
  \begin{itemize}
      \item Ogni wallet è una coppia $(d,Q)$ + identificatore address.
      \item Il possesso di $d$ abilita firma e quindi controllo logico dei fondi associati all'address.
  \end{itemize}
  
  \item Definiamo lo stato iniziale $\Sigma_0: \text{address} \mapsto (balance, nonce)$.
  \begin{itemize}
      \item È il riferimento condiviso per la validazione deterministica tra nodi.
  \end{itemize}
\end{enumerate}

\vfill
\begin{center}
\begin{minipage}{0.7\textwidth}
\begin{alertblock}{\textbf{\faLaptop\ Implementazione}}
Generazione del Wallet e dello Stato Iniziale
\end{alertblock}
\end{minipage}
\end{center}
\end{frame}



\begin{exerciseFrame}{Esempio della creazione di un Wallet su $\mathbb{F}_{17}$ [E1]}

\small 
\textbf{Curva ellittica} $E:\; y^2 \equiv x^3 + 2x + 2 \pmod{17}.$

\textbf{Gruppo} $E(\mathbb{F}_{17}) \text{ ha } 19 \text{ punti (incluso } \mathcal{O}\text{)} \Rightarrow n=19.$

\textbf{Generatore scelto:} $G=(0,6), \quad \operatorname{ord}(G)=19.$

\textbf{Chiavi:} $d=7,\quad Q=dG=7G.$

\vspace{0.2em}
\hrule
\vspace{0.3em}

Somma e Differenza sono definite come segue:

Per $P\neq Q$:
$
\lambda=\frac{y_Q-y_P}{x_Q-x_P}\ (\mathrm{mod}\ 17),\quad
x_{P+Q}=\lambda^2-x_P-x_Q,\quad
y_{P+Q}=\lambda(x_P-x_{P+Q})-y_P.
$

Per $2P$: $\lambda=\frac{3x_P^2+a}{2y_P}\ (\mathrm{mod}\ 17).$

\vspace{0.3em}
\hrule
\vspace{0.2em}

Calcolo di $2G$: 
$\lambda=\frac{3\cdot 0^2+2}{2\cdot 6}=\frac{2}{12}\equiv 2\cdot 12^{-1}\equiv 2\cdot 10\equiv 3\ (\mathrm{mod}\ 17)$

$x_{2G}=3^2-0-0=9,\qquad y_{2G}=3(0-9)-6=-33\equiv 1\ (\mathrm{mod}\ 17) \to 2G=(9,1).$


Iterando la somma otteniamo: \quad $7G=(13,10)=Q$.

\vspace{0.3em}
\hrule
\vspace{0.3em}

Definiamo una \textbf{funzione di hash} didattica:
$H(x,y)=x+2y\ (\mathrm{mod}\ 17),\qquad $

Otteniamo $H(Q)=H(13,10)=13+2\cdot 10=33\equiv 16.$

\vspace{0.3em}
\hrule
\vspace{0.3em}

Abbiamo quindi generato il nostro \textbf{wallet:}\quad $(d,Q,H(Q))=\bigl(7,(13,10),16\bigr)$.
\end{exerciseFrame}



\begin{frame}{Tre nonce da distinguere}
\begin{block}{Nonce crittografico $k$}
Usato nella firma ECDSA. \textbf{Se riusato}, può esporre $d$.
\end{block}
\begin{block}{Nonce account $n_{acc}$}
Contatore per account nella blockchain (anti replay/ordine tx).
\end{block}
\begin{block}{Nonce mining $n_{pow}$}
Valore variabile nel blocco per soddisfare PoW: hash con prefisso di zeri.
\end{block}
\end{frame}

\begin{frame}{Transazione come cambiamento di stato autorizzato}

\small

\begin{block}{Idea chiave}
\textbf{Transazione = messaggio firmato} che autorizza un cambiamento deterministico
dello \textbf{stato globale}.
\end{block}


\textbf{1) Payload (azione richiesta)}
\[
\{"from": A,\ "to": B,\ "value": 5,\ "nonce": n_{acc}\}
\]


\textbf{2) Digest}
\begin{itemize}
  \item hash crittografico:
  \[
    z = H(\texttt{payload}) \bmod n
  \]
\end{itemize}


\textbf{3) Effetto}
\[
\Sigma \;\longmapsto\; \Sigma'
\qquad
\]

\end{frame}


\begin{frame}{Firma ECDSA e autorizzazione del cambiamento di stato}

\small

\textbf{1) Firma ECDSA}
\[
R = kG, \quad r = x_R \bmod n, \quad
s = k^{-1}(z + rd) \bmod n
\]
\begin{itemize}
  \item firma $(r,s)$ = contenuto + identità del mittente
\end{itemize}


\textbf{2) Verifica}
\begin{itemize}
  \item verifica regole (nonce, fondi, formato)
  \item verifica firma (chiave pubblica $Q$)
\end{itemize}
\[
u_1 = zs^{-1},\quad u_2 = rs^{-1},\quad
X = u_1G + u_2Q
\]

dove $X$ è il punto ricostruito in fase di verifica che, in caso di firma valida, coincide con $R$.

Una transazione valida viene inserita nella \textbf{mempool}


\vfill
\hrule


\textbf{Nota di sicurezza:}
$k$ deve essere segreto e unico

\end{frame}

\begin{exerciseFrame}{Esempio di Firma ECDSA [E2]}

\small

\textbf{Setup (come in E1):}


$E:\; y^2 \equiv x^3 + 2x + 2 \pmod{17}, \quad G=(0,6), \quad n=19$

$d=7, \quad Q=(13,10)$

\vspace{0.3em}
\hrule
\vspace{0.3em}

\textbf{Messaggio:} $z = 11 \in \mathbb{Z}_{19}$ (digest della transazione)

\textbf{Scelta del nonce crittografico:} $k=5$

\vspace{0.3em}
\hrule
\vspace{0.3em}

\textbf{Calcolo:}\\
\[
R = kG = 5G = (10,11) \to r = x_R \bmod n = 10
\]

\[
k^{-1} \equiv 5^{-1} \equiv 4 \pmod{19}
\]

\[
s = k^{-1}(z + rd) \bmod n = 4(11 + 10\cdot 7) \bmod 19 = 1
\]

\vspace{0.3em}
\hrule
\vspace{0.3em}

Quindi abbiamo la \textbf{Firma ECDSA:} $(r,s) = (10,1)$



\end{exerciseFrame}




\begin{frame}{Come avviene il Mining di un blocco?}
\textbf{Lato matematico/protocollo:}
\begin{itemize}
  \item Si costruisce header del blocco con hash precedente, tx, difficoltà, proposer.
  \item La \textbf{difficulty} (nel progetto default $d=2$) impone quanti zeri esadecimali iniziali deve avere l'hash del blocco.
  \item Si cerca $n_{pow}$ tale che:
  \[
    \mathrm{H_{SHA}}(\text{header}, n_{pow}) = h, \quad h \text{ inizia con } d \text{ zeri esadecimali.}
  \]
  \item Probabilità di successo per tentativo: $\approx 16^{-d}$ (hash uniforme).
  \item Tentativi attesi: $\approx 16^d$ (con $d=2$, circa $256$ tentativi medi).
  \item Una volta trovato, il blocco è candidato valido PoW.
\end{itemize}

\end{frame}


\begin{frame}{Creazione Transazione e Mining}
\textbf{Pipeline concettuale:}
\begin{enumerate}
  \item Costruisco payload canonico (JSON).
  \item Calcolo digest $z = H(payload)$.
  \item Calcolo la firma ECDSA $\Rightarrow (r,s)$.
  \item \textbf{Autenticità Crittografica}: Nodo verifica firma + regole (indirizzo, fondi, nonce).
  \item \textbf{Validità di Protocollo}: Se valida, tx entra in mempool e viene propagata ai peer.
  \item Il miner prende l'intero blocco e cerca l'$n_{pow}$ che verifica la condizione. Se ci riesce viene ricompensato.
\end{enumerate}

\vfill
\begin{center}
\begin{minipage}{0.7\textwidth}
\begin{alertblock}{\textbf{\faLaptop\ Implementazione}}
Invio Transazioni e Mining
\end{alertblock}
\end{minipage}
\end{center}

\end{frame}





\section{Esperimento 2: Attacco Weak Nonce Reuse}
\begin{frame}{Problema matematico: riuso dello stesso $k$}
Per due messaggi diversi $z_1, z_2$ firmati con stesso $k$:
\[
s_1 = k^{-1}(z_1 + rd), \qquad s_2 = k^{-1}(z_2 + rd).
\]
Sottraendo:
\[
k = (z_1 - z_2)(s_1 - s_2)^{-1} \bmod n.
\]
Poi:
\[
d = (s_1k - z_1)r^{-1} \bmod n.
\]
\textbf{Conclusione:} due firme con stesso $r$ (quindi stesso $k$) bastano a recuperare la chiave privata.
\end{frame}

\begin{frame}{Attacco ad una Weak Transaction}
\textbf{Pipeline vulnerabile:}
\begin{enumerate}
  \item Genera un $k_0$ casuale una sola volta.
  \item Crea tx1 con nonce account corrente, firma con $k_0$, invia.
  \item Mina subito un blocco (tx1 viene confermata, nonce account incrementa).
  \item Crea tx2 con \textbf{nuovo nonce account} ma stesso $k_0$, invia.
\end{enumerate}

\textbf{Pipeline attaccante:}
\begin{enumerate}
  \item Legge stato chain + mempool via API nodo.
  \item Scansiona tutte le tx e raggruppa per coppia (pubkey, $r$).
  \item Se trova due tx con stesso $r$, applica formule di recovery di $k$ e $d$.ù
\end{enumerate}

\vfill
\begin{center}
\begin{minipage}{0.7\textwidth}
\begin{alertblock}{\textbf{\faLaptop\ Implementazione}}
Attacco ad una Weak Transaction
\end{alertblock}
\end{minipage}
\end{center}

\end{frame}

\begin{exerciseFrame}{Esempio di Riuso del nonce ECDSA (E3)}

\small

\textbf{Scenario (come in E1 e E2):}
\[
E:\; y^2 \equiv x^3 + 2x + 2 \pmod{17}, \qquad
G=(0,6), \qquad n=19
\]
\[
d=7, \qquad Q=(13,10)
\]

\textbf{Prima transazione:}
\[
z_1=11, \quad k=5 \quad \Longrightarrow \quad (r,s_1)=(10,1)
\]

\textbf{Seconda transazione (nonce duplicato):}
\[
z_2=3, \quad k=5 \quad \Longrightarrow \quad (r,s_2)=(10,7)
\]

\hrule
\vspace{0.4em}

\textbf{Recovery del nonce:}
\[
k = (z_1 - z_2)(s_1 - s_2)^{-1} \bmod n
\]
\[
z_1-z_2 = 8, \quad s_1-s_2 = -6 \equiv 13, \quad 13^{-1}\equiv 3
\]
\[
k = 8 \cdot 3 \bmod 19 = 5
\]

\textbf{Recovery della chiave privata:}
\[
d = (s_1k - z_1) r^{-1} \bmod n
\]
\[
s_1k - z_1 = 5 - 11 = -6 \equiv 13, \quad r^{-1} = 10^{-1} \equiv 2
\]
\[
d = 13 \cdot 2 \bmod 19 = 7
\]

\begin{alertblock}{Conclusione}
Il riuso del nonce $k$ in ECDSA consente il \textbf{recupero completo della chiave privata}.
\end{alertblock}

\end{exerciseFrame}

\begin{frame}{Conclusione}

\vfill

\begin{center}
{\LARGE \textbf{Grazie per l’attenzione}}

\vspace{0.5cm}

{\large
Michele Magrini \hspace{1em} (matr. \texttt{2066963})\\[0.4em]
Francesco Marrocco \hspace{1em} (matr. \texttt{2062223})\\[0.4em]
Leo Vincenzo Petrarca \hspace{1em} (matr. \texttt{2087113})
}


\vspace{0.6cm}

\qrcode[height=3cm]{https://github.com/mich1803/ECDSA-Blockchain}

\vspace{0.2cm}

\textbf{\href{https://github.com/mich1803/ECDSA-Blockchain}{Repository GitHub}}

\end{center}

\vfill

\end{frame}



\end{document}